\documentclass[fleqn,12pt]{olplainarticle}
% Use option lineno for line numbers 

\title{Internal Migration of Mexican Nationals in the United States}

\author[1]{Neal Marquez}
\author[1]{Sara Curran}
\affil[1]{Sociology Department, University Washington}

\keywords{Internal Migration, Mexican Nationals, United States}

\begin{abstract}
Please provide an abstract of no more than 300 words. Your abstract should explain the main contributions of your article, and should not contain any material that is not included in the main text. 
\end{abstract}

\begin{document}

\flushbottom
\maketitle
\thispagestyle{empty}

\section*{Introduction}

% restructure introduction
% Studies of internal migration have largely focused on native born populations and the influence of economy driving flows, 
% studies focusing on economic, rural/urban, and black/white
% Ellis, Lichter, Townley
% separate silo of work has focused on international migration 
% and distinction between trad destinations and new/destinations
% Singer Massey
% Not as much work has been done on the intersection of FB pops and their internal migration 
% This is of great importance for the foreign born Mexican population
% people will talk about FB pops and Hispanic pops but their are important distinctions between these groups.

Studies of internal migration in the United States are considerably less common than studies of international migration.
In the field of Sociology in particular, a majority of the theorization of emigration, immigration, and integration are directed towards the international migration context, and are tested in such situations \citep{Massey1993, Luthra}.
This should not come as a surprise, as Socoiolgy has largely focused on issues of Migrant behavior in relation to social structures such as networks, enclaves, and social capital \citep{Brettell2007}.
The mechanisms that create "social distance", as defined by \cite{Bean2015}, seem to be less at play for the internal migration context where language, culture, and norms have a much larger overlap between sending and receiving locations.

The distinction between international and internal migration, however, has been called into question \citep{Ellis2012, King2010}.
Studies of internal migration in the United States have largely focused on the general population \citep{Molloy2011a, Cooke2013} or have concentrated largely on a set of repeatedly studied factors such as economic drivers of immigration \citep{Lichter2009, Light2009, Kritz2010, Calnan2017}, rural/urban transitions of populations \citep{Hernandez-Leon2000, Kandel2005}, or black/white migration in the Great Migration era \citep{Eichenlaub2010}.
However in the case of the migration of the foreign born population, theories of migration from the international migration field have more relevance \citep{King2010}.
Within the United States the number of individuals who are international migrants is less than half of the number of internal migrants who have left there home state, according to the most recent US census.
Of particular interest to sociologists is the case of internal migration of the foreign born population.   
Again looking at the context of the United States, a large body of research has focused on the migration rate, destination, and experiences of Mexican immigrants in their new context.
Their experiences, however, are often analyzed in the "Mexican in the United States" lens, often omitting the pathways that have led individuals to their new home destination.
Though there has been a large body of research on the "dispersion" of the Mexican population in the United States, the pathways by which migrants make it to these new destinations is often ignored, especially in quantitative analysis \citep{Light2009, Massey2010, Crowley2010, Johnson2016}. 

This paper looks to expand on the literature of internal migration and diffusion of the Mexican born population living in the United States (MBPUS).
Where other articles have predominately focused on growth of the Mexican population within particular geographies, or have analyzed internal flows of Mexican populations in an isolated context, we seek to study how flows of the MBPUS differ from other populations.  
In addition, we present a number of possible theories that could be driving internal migration of the MBPUS from both internal and international migration literature and why we theorize that this populations patterns of internal migration may differs from other populations.

\section*{Background}

\subsection*{Mexican and Hispanic Population and Migration}

Since 1970, the growth of the Mexican born population in the United States has been more than 10 fold, moving from less than 1 million in the total population to over 11 million in recent years \citep{ Gonzalez-Barrera2013, Noe-Bustamante2019a}.
Similarly, there has been an abundance of soico-demographic research monitoring, estimating the total of, and understanding the drivers of migration from Mexico to the United States \citep{Massey1986}.
For a subset of researchers, the particular geographies in the United States that have drawn Mexican nationals is of great importance.
While historically, states such as California, Texas, and Illinois have been home to the largest number of MBPUS, a number of studies have found that "new destinations" have seen much higher rates of growth in recent years such as in the South and South East \citep{Singer2004, Ellis2016,  Hernandez-Leon2000,  Massey2008,  Garip2016,  Crowley2006, Massey2002}.
While the driving forces for the change in where Mexican nationals locate in the United States is hotly discussed, less focus is placed on the pathways that lead to migrants to these new destinations which has been overwhelmingly from internal migration movement rather than direct migration from Mexico \citep{Lichter2009}. 

Though not explicitly about internal migration, in the work of \cite{Massey2002} the authors describe what they see as the driving force of the growth in new destinations.
This is namely the effect of the Immigration Reform and Control Act 1986 (IRCA).
Massey et al. argued that increases in border protection and enforcement along traditional locations near San Diego and Texas diverted migration away from those locations and created new pathways of migration with new end destinations.
This is further evidenced by changes observed in the composition of the Mexican population in different geographies in other studies.
Namely \citep{Riosmena2012} found that the geographic origin within Mexico of the MBPUS is compositionally different depending on the geography within the United States.
In later writings Massey expanded on the possibility of other factors that could potentially lead to growth in new destinations \citep{Massey2008}.
Namely he allowed for factors that could drive internal migration of the MBPUS, such as hostile laws and changes in the labor market, to account for shifts to new destinations, but ultimately still attributed a bulk of the growth to changes in the patrolling of the border.  
Several, albeit few, studies have directly analyzed how internal migration of the MBPUS is altered by changes in internal geographic factors.
\cite{Ellis2016} focused on changes in state legal practices directed at undocumented immigrants and the response in terms of migration to and from those states.
Following the construction of \cite{Leerkes2012}, states were labeled as hostile towards immigrants if they had employer participation in E-Verify, restrictive state laws towards immigrants, or had county and city involvement in the 287G program.
Looking not only at the Hispanic immigrant population but also the native Hispanic population Ellis at al found that the passing of anti-immigrant laws decreased the probability of moving towards those states.
However, changes in laws directed at tracking undocumented populations did not account for a significant amount of out-migration from these states by Hispanic immigrant populations after controlling for a number of economic characteristics.
This may be in part due to the variation in the policies and enforcement across states as earlier studies found that Hispanic immigrants were more likely to emigrate from Arizona after the passing of the Legal Arizona Workers Act \citep{Ellis2014a}.
These findings seem to be in favor of the assertions made by Massey et al that changes in IRCA led to a redirection of the migration stream rather than a redistribution of the current population in the United States.

Perhaps in the most direct study of internal migration of the Mexican born population in the United States, \cite{Lichter2009} studied how population growth of Hispanic communities in new destinations of the United States was driven in large part by internal migration from traditional areas.
Building off the work of \cite{Hernandez-Leon2000}, \cite{Singer2004}, and \cite{Kandel2005}, Lichter and Johnson looked to better characterize the pathway and factors that led to growth in new destinations.
Using Consistent Public Use Micro Areas (C-PUMAs) as their units of analysis they sought to define each geography as either an established, high growth, or other, based on their historical Hispanic population numbers, and examine movement between geographies. 
The authors found that growth in emerging geographies, largely situated in what other authors refer to as new destinations, was almost equal parts growth from movement across the US border and internal migration from one geography to another.
Additionally, they found that migration rates differed by education levels and foreign-born status with more educated foreign born Hispanic more likely to emigrate from their current geography. 

The focus on social capital variables driving migration is a staple in the way economists have studied changes in the geographic distribution of the MBPUS.
\cite{Card2005} found that 85\% of growth in new destinations could be attributed to economic variables rather than political variables as speculated by many others.
\cite{Light2009} further backed this argument by stating although network driven migration played a large part in earlier years of Mexican immigration, the importance diminished over time as the rent to wage ratio shifted in an unfavorable manner as immigrant ethnic populations increased.
Both of these articles, did not distinguish from growth due to external vs internal migration and assumed Mexican foreign born individuals to have a similar process regardless of their place of emigration.
In addition, their study variable is only growth of the Mexican born population which, although can only have growth for immigration, can decline for emigration and death and does not account for the total migration amount but rather only the net migration.
Furthermore how this migration relates to immigration of Hispanics in the United States who are descendants, relatives, and co-ethnics of the MBPUS is not made clear.

\subsection*{Theories of International and Internal Migration}

Internal migration in the United States has seen a stronger focus in recent years largely because of declines at all levels of internal migration in the general population  \citep{Molloy2011a}.
In a 2011 paper by Molly et al, the researchers found that migration has been on the decline since the 1980's.
While there have been periods of temporary increase the general trend of decline has persisted across a number of levels of migration, that is the minimum distance required to constitute a move to a new location.
They additionally found that foreign born individuals were less likely to have recently migrated than their native born counter parts across all years of the study.
Other studies have placed a more direct focus on the internal migration patters of the foreign born population.
Lichter et al described in one paper the expansion of MBPUS into rural locations of the United States  \citep{Lichter2012a}.
While they did not specifically look at rates of internal migration, they do allude to the fact that that their is likely a diffusion process of the MBPUS population from traditional locations and is corroborated by other studies  \citep{Lichter2009, Frey2011}.
Other studies have dug into the mechanisms that drive migration to these new destinations  \citep{ Kandel2005,  Ellis2016,  Hernandez-Leon2000}, however, it is difficult to differentiate if these newcomers are MBPUS who have previously lived in the United States or whether they are coming from Mexico.
In a recent study several authors found that patterns of internal migration of the foreign population could be well observed using IRS records  \citep{Foster2018} which could lend insight to differentiating recent arrivals from those MBPUS who had previously been in the United States. 

The general drivers of migration are many, however, studies of internal migration tend to use a model which focuses on individual or family level decision making. 
In this model families or "individuals choose consumption and location to maximize utility given the prevailing wage and price level in each location"  \citep{ Molloy2011a}.
Migration is seen as a utility maximization process by which individuals are continuously seeking the best location to move in order to meet the economic and social needs of them and/or their family.
Generalizations of this model have been incorporated into previously mentioned economic studies of Hispanic migration \citep{Light2009, Card2005, Lichter2009}.
This model, however, neglects to account for the different socio-political local contexts that may drive migration, especially in the sending location.
It is likely that individuals only entertain the idea of migration if is a perceivable option.
In the international migration context, Mexico has been historically in a similar socio-demographic state as other Latin American countries where migration presented an opportunity for greater social and economic positioning, however, it is likely that the proximity and political relationship to the United States the flow of migrants from Mexico to the United States \citep{Massey2009}. 
Furthermore a number of other cultural and political factors, laws that either attack or protect migrants  \citep{Ellis2016}, outside of the individuals of the family and individual would effect where individuals are likely to migrate and likely have a differential effect on the Mexican born population in the United States compared to other foreign born groups.
Lastly, we would expect that the factors that effect migration would deferentially effect the MBPUS population compared to other populations in the United States.
MBPUS occupy a different socio-economic position than other Foreign Born Immigrants and native born Hispanics in the United States that would likely have an impact on their rates, locations, and flows of internal migration.  

In addition, we expect their to be a number of varying factors across the United States which would alter the rate of internal migration of MBPUS.
Again, previous studies have shown that recent migration varies by regions in the United States \citep{ Molloy2011a} and how this may effect the foreign born population, \citep{Ellis2016, Lichter2009}, however, how these patterns are altered specifically for MBPUS is not well studied. 
We state four mechanisms that are likely to alter the rates of internal migration of MBPUS in different regions of the United States. 
1) Different regions present different economic and social opportunities. This hypothesis falls in line with the standard model of migration mentioned earlier. Individuals are morel likely to migrate to places if they see an opportunity that better presents itself in a new location  \citep{ Ellis2014}.
This measure is difficult to quantify because it requires information about both the sending and receiving location in terms of their economic differences for both individuals who did and did not migrate. 
2) The local context of reception to immigrants alters where individuals are moving displacing some individuals via chilling mechanisms and welcoming others through protective measures.
Other authors have argued on the importance of local context acting as a deterrent  \citep{ Ellis2016} or incentive  \citep{Jaworsky2012} for foreign born populations in the United States which could act as pull or push factors for migration. 
3) The demographics of the MBPUS population in the United States vary by region in important ways which alter the rates of internal migration. Previous scholars have written on the changing demographics of MBPUS  \citep{ Garip2016,  Riosmena2012}, dismantling the idea that the Mexican immigrant is well fit by one description. 
Changes in the level of education, previous work experience, sex ratio, and geography of MBPUS have changed overtime  \citep{ Garip2016} and have corresponded to periods of differential growth of the MBPUS population in different locations in the United States.
It may very well be that the differences in geography for recent re-settlement of the MBPUS population may be due to demographic differences.
4) Networks link locations together and and facilitate flows of internal migrants in a similar fashion to theories of international migration.

In this paper we analyze rates of internal migration of the MBPUS.
We hypothesize the rates of internal migration for the MBPUS population are different than the majority population, native born whites, other immigrant populations, as well as native Hispanic populations.
While we do not explicitly test for the factors that are change the patterns of internal migration of MBPUS, we expect that the relationships are different than what previous studies will have found because of the 
In addition, we suspect that local contexts alter rates of migration for these population in different ways, changing the relationship of who may migrate most often and giving importance to the intermediary steps of migration within the United States.
Lastly we suspect that while demographic factors that are important to internal migration do indeed effect rates of migration for these groups, the demographic differences should not explain the overall differences between each of these groups.  

\section*{Data and Methods}

In order to asses rates of migration for different immigrant and ethnic groups we will use Public Use Microdata from the American Community Survey (ACS) provided by IPUMS USA.
IPUMS USA provides a consistent set of variables across multiple survey years in order to assess changes over time.
We use the ACS 1 year data from 2012, 2015 and 2018 (the most recently available year of data). 
We begin our analysis in 2012 as this is the first year that uses the consistent geographic identifier of importance in our analysis, the Migration PUblic Use Micro Area (M-PUMA).
M-PUMA are geographically consistent and comprehensive areas used over a period of time by the US Census which are made up of 1 or more commonly used PUMA.
The current series of M-PUMA are consistent over the years 2012-2021 and unlike metropolitan areas which are commonly used in migration studies, are comprehensive of the entire land mass of the United States.
This distinction is important as we would like to pay close attention to more rural areas that are not captured by the metropolitan area geographies.
In addition, M-PUMA are reported for all individuals surveyed by the ACS at both the time of the survey as well as one year prior to the survey, unlike county level identifiers.
While MPUMA are not typical metropolitan areas or government jurisdictions they are consistent over the period of analysis and provide more geographic resolution than state level analysis.
The data provided tell us if a move was made between MPUMA in the last year prior to the survey and allow us to track movement flows from one location to another, unlike other studies which only focus on net growth, even when net migration is near zero. 

We also consider a number of other variables to adjust for the analysis of migration flows between locations. The variables collected on each individual are as follows age, sex, race, Hispanic origin, foreign born status, and place of birth. We restrict our analysis to the lower 48 contiguous states as only an extremely small portion of the MBPUS population reside in Alaska or Hawaii ($<.1\%$).

We hypothesize that different geographic regions will have different effects on the rates of internal migration of MBPUS. We place each MPUMA into one of 7 regions of the United States which are Border Land, NorthWest, Great Plains, South, SouthEast, NorthEast, and Great Lakes. These areas represent different historical areas of migration of MBPUS in the United States and have been defined in previous research  \citep{ Riosmena2012}.  We classify anyone declares them self as any form of Hispanic to be Hispanic, no matter their classification for race, as long as they were born in the United States. Individuals born outside of the Untied States were classified as either "Mexico" if born in Mexico, or "Other Migrant" if born elsewhere. We also include native born Whites in the analysis as additional reference group. Individuals for whom did not live in the United States in the year prior to the analysis were removed from the study. In addition, we remove the small amount of individuals who either would not disclose internal migration behavior (that is migration from one MPUMA to another) in the past year, or for who their status of internal migration was unknown. This process remove less than $<1\%$ of individuals from our analysis.

We provide estimates of the probability of MPUMA migration at a number of levels of aggregation to verify that there is indeed variation between our groups of interest. Uncertainty for all estimates are calculated using Horvitz-Thompson estimators from the sampling weights provided by the ACS. For summary statistics we aggregate ages into five year age bins in order to get more stable estimators. We then run a number of survey regressions with survey weights in order to asses which sets of variables best explain the process of migration. Our regressions are as follow and represent increasing complexity in the migration process. 1) A model with only dummy variables for year of the survey. 2) Model 1 with additional sex dummy variable. 3) Model 2 with ethnicity and nativity groups. 4) Model 3 with dummy variable on location of residence of individuals into one of the 7 previously mentioned regions. 5) Model 4 with and interaction effect between dummy regions and ethnic/nativity group. 6) Model 5 with additional quadratic age effect. 7) Model 6 with year of survey dummy, ethnic/nativity group, and region three way interaction effect. 8) Model 6 with age, ethnic/nativity group, and region three way interaction effect. 9) Model 6 with sex, age, ethnic/nativity group, and region four way interaction effect. 10) Model 8 with sex and ethnic/nativity group interaction effect. Each model represents an increase in complexity from the previous model until we build up to our final model 10 which is our theoretical model of choice. In order to asses model performance we use a modification of the BIC which is suited for survey glms.  

\section*{Results}


\bibliography{Hispanic-Migration,Migration,Demography}



\end{document}